\section{Motivation}
The process of learning words usually begins before a human reaches the age of one. At around 18 months, the average child is learning words at a rate of one every two waking hours (\cite{pinker1995language,clark1995lexicon}). This rate of learning can be maintained and even increased moving into adolescence. Adults have a tough time reaching those numbers and face a lot more difficulty when learning a second language. Adults need to spend more time learning words as well as make a conscious effort to do so. In adults, words are gradually
learned over a period of time from numerous exposures. (\cite{schmitt2000vocabulary}). My motivation to undertake this project stems from my difficulties in retaining the meaning of words after I looked them up. As for why one should seek to improve their vocabulary in the first place, it is worth consulting George Orwell's `Politics and the English language' (\cite{orwell1968politics}). In what is perhaps his most famous essay, Orwell articulates the idea that there is a correlation between the quality of our language and the quality of our thoughts. Words are indispensable tools for our thoughts, and so it follows that appending to this tool-set will only serve to better oneself, as well as society as a whole. 


\section{Existing Applications}

The most common methods for learning words are:

\begin{itemize}
    \item Use of a dictionary to learn the definition.
    \item Trying to find similar or identical words. This includes use of a thesaurus or translator.
    \item Interpreting the meaning using context around the word.
    \item A combination of any of the above.
\end{itemize}


Online dictionary's are incredibly useful, but they don't accommodate for the `numerous exposures' requirement. They don't encourage the user to review the words that are queried, and they often provide weak example sentences that don't reveal much about the word.\\ 

There do exist sites that allow the user to save words and be tested on them. However, these websites also tend to have irrelevant example sentences and are very reliant on synonyms. They use synonyms to test the user's knowledge of a word. For example, if a user were being tested on the word `Election', they would be shown four words, one of which is what they believe to be a synonym for `Election'. There are a couple of problems with this--- the first being that the user may not know all of the synonyms they are being shown. The second problem is that these are synonyms are usually weak in the respect that you can't use them in the same context as the original word. 
\\

This application will use sets of \emph{synsets} found on WordNet\textsuperscript{\textregistered} (\cite{miller1995wordnet}), a dictionary created by the University of Princeton ``A synset is a group of words that are synonymous, in the
sense that there are contexts in which they can be interchanged
without changing the meaning of the statement (\cite{miller1993semantic}).
\\

Commonplace in all dictionary's and services are example sentences that aren't at all curated to be `useful'. Vocabulary.com tries to find recent usages of the word on news websites. This only offers a chance that the sentence will include useful contextual information about the word. An example sentence is `useful' if the context provides information about the word in question. For example, `London Buses have a distinct \emph{red} colour' is  much more useful than `My box is \emph{red}' as an example sentence for \emph{red}. A good example example sentence can reveal the meaning of a word. Accordingly, a large proportion of this project will be attempting to create a filter for `good' and `bad' example sentences.