\section{Front End}
The front-end will take the form of an app. I have a slightly optimistic goal of making it cross platform --- that is --- building an application that will work on both Android and iOS. To achieve this, I will use React Native, which is a Javascript framework for creating native mobile apps (\cite{reactNative}). I will use this in combination with (\cite{Expo}), which is a ``free and open source toolchain built around React Native''. Developing iOS apps typically require use of Xcode, which is a software only found on iMacs. Expo allows you to test on an iPhone regardless of your computer's operating system. This brings the advantage of being able to release the app to a wider audience.

\section{Back-end}

The back-end will revolve around a REST controller. Restful API's make use of intuitive HTTP verbs such as GET, POST, PUT, DELETE to perform actions on entities (\cite{richardson2008restful}). Such entities --- in the context of the app --- would include users, words and sentences. The controller will be written in Kotlin, a language which runs on the JVM. The controller will link to a MongoDB, which is a noSQL document-based database. There will exist a document for each user, which will take the form of a JSON. Inside this JSON, various variables will be stored such as the words the user has looked up, how many words they have learned, and so on.\\

The back-end will also need to integrate with various API's. These include Oxford Dictionary API and WordNet\textsuperscript{\textregistered} (\cite{miller1995wordnet}), which must be used every time a word is queried. The sentence-sorting component also comes under the umbrella term `back-end'. The source for the sentences will come from Oxford Dictionary API, as well as text files of open source books. I will then use  Apache (\cite{apache}) to perform Natural Language Processing (NLP) tasks on the texts. This will include the code to identify frequent word associations. As of now, I am unsure whether or not I will also obtain data in real time from search engines. This would involve creating a web-crawler, which I would do using Kotlin. The considerations for this approach are included in the next section of the report.

\section{Project Management and VCS}

I have been using project-management tool `Asana' to track tasks and progress (\cite{asana}).


